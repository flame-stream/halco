%%% Tables
\usepackage{array,tabularx,tabulary,booktabs}
\usepackage{longtable}
\usepackage{multirow}

%%% Images
\usepackage{graphicx}
\graphicspath{{images/}{images2/}}
\setlength\fboxsep{3pt}
\setlength\fboxrule{1pt}
\usepackage{wrapfig}
\usepackage{tikz}
\usepackage{pgfplots}
\usepackage{pgfplotstable}

%%% AMS
\usepackage{amsmath,amsfonts,amssymb,amsthm,mathtools}
\usepackage{icomma}

%%% Fields
\usepackage{geometry}
% \geometry{top=25mm}
% \geometry{bottom=35mm}
% \geometry{left=35mm}
% \geometry{right=20mm}

%%% Making the bibliography appear in the table of contents
%%% https://tex.stackexchange.com/questions/8458/making-the-bibliography-appear-in-the-table-of-contents
\usepackage[nottoc,numbib]{tocbibind}

%%% Diff
\usepackage{cmap} % Search in PDF
\usepackage{mathtext}
\usepackage{xspace}
\usepackage{etoolbox} % Logical operators
\usepackage{listings}
\usepackage{color}
\usepackage{lastpage}
\usepackage{soul}
\usepackage{hyperref}
%\usepackage[usenames,dvipsnames,svgnames,table,rgb]{xcolor}
\usepackage{csquotes}
\usepackage{multicol}
\usepackage{array}
\usepackage{setspace}
\newcolumntype{L}[1]{>{\raggedright\let\newline\\\arraybackslash\hspace{0pt}}m{#1}}
\newcolumntype{C}[1]{>{\centering\let\newline\\\arraybackslash\hspace{0pt}}m{#1}}
\newcolumntype{R}[1]{>{\raggedleft\let\newline\\\arraybackslash\hspace{0pt}}m{#1}}

\definecolor{mygreen}{rgb}{0,0.6,0}
\definecolor{mygray}{rgb}{0.5,0.5,0.5}
\definecolor{mymauve}{rgb}{0.58,0,0.82}

\lstset{
  frame=none,
  xleftmargin=2pt,
  stepnumber=1,
  numbers=left,
  numbersep=5pt,
  numberstyle=\ttfamily\scriptsize\color[gray]{0.3},
  belowcaptionskip=\bigskipamount,
  captionpos=b,
  escapeinside={*'}{'*},
  language=haskell,
  tabsize=2,
  emphstyle={\bf},
  commentstyle=\it,
  stringstyle=\mdseries\rmfamily,
  showspaces=false,
  keywordstyle=\bfseries\rmfamily,
%   columns=flexible,
  basicstyle=\small\ttfamily,
  showstringspaces=false,
  morecomment=[l]\%,
  breaklines,
  columns=fullflexible,
  flexiblecolumns,
  numbers=left,
  numberstyle={\footnotesize},
  extendedchars=\true,
  keepspaces=true,
}

% \lstset{
%   backgroundcolor=\color{white},   % choose the background color; you must add \usepackage{color} or \usepackage{xcolor}; should come as last argument
%   basicstyle=\footnotesize,        % the size of the fonts that are used for the code
%   breakatwhitespace=false,         % sets if automatic breaks should only happen at whitespace
%   %breaklines=true,                 % sets automatic line breaking
%   captionpos=с,                    % sets the caption-position to bottom
%   commentstyle=\color{mygreen},    % comment style
%   deletekeywords={...},            % if you want to delete keywords from the given language
%   escapeinside={\%*}{*)},          % if you want to add LaTeX within your code
%   extendedchars=true,              % lets you use non-ASCII characters; for 8-bits encodings only, does not work with UTF-8
%   %firstnumber=1000,                % start line enumeration with line 1000
%   %frame=single,	                   % adds a frame around the code
%   keepspaces=true,                 % keeps spaces in text, useful for keeping indentation of code (possibly needs columns=flexible)
%   %keywordstyle=\color{blue},       % keyword style
%   %language=SQL,                 % the language of the code
%   %morekeywords={*,...},            % if you want to add more keywords to the set
%   %numbers=left,                    % where to put the line-numbers; possible values are (none, left, right)
%   %numbersep=5pt,                   % how far the line-numbers are from the code
%   %numberstyle=\tiny\color{mygray}, % the style that is used for the line-numbers
%   rulecolor=\color{black},         % if not set, the frame-color may be changed on line-breaks within not-black text (e.g. comments (green here))
%   showspaces=false,                % show spaces everywhere adding particular underscores; it overrides 'showstringspaces'
%   showstringspaces=false,          % underline spaces within strings only
%   showtabs=false,                  % show tabs within strings adding particular underscores
%   stepnumber=2,                    % the step between two line-numbers. If it's 1, each line will be numbered
%   stringstyle=\color{mymauve},     % string literal style
%   tabsize=2,	                   % sets default tabsize to 2 spaces
%   title=\lstname                   % show the filename of files included with \lstinputlisting; also try caption instead of title
% }

\mathtoolsset{showonlyrefs=true} % Numerete only formules with references

\newcommand*{\yooline}[1]{\overline{\overline{#1}}}
\newcommand*{\yeqdef}{\overset{\underset{\mathrm{def}}{}}{=}}
\newcommand*{\Pn}{\ensuremath{\mathbf{P}^n}}
\newcommand*{\R}{\ensuremath{\mathbb R}\xspace}
\newcommand*{\Laplace}{\mathop{}\!\mathbin\bigtriangleup}
\newcommand*{\DAlambert}{\mathop{}\!\mathbin\Box}
\DeclareMathOperator{\sgn}{\mathop{sgn}}

%% VsCode liter suggested to add
\pgfplotsset{compat=1.17}
