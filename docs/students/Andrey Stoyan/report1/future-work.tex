\section{Дальнейшая работа}

Несмотря на объём проделанной работы, реализуемость идеи ещё не доказана:
\begin{itemize}
    \setlength\itemsep{-0.2em}
    \item Не очевидно существование достаточно быстрого алгоритма перебора всех конкретных графов, задаваемых данным ц-графом. Ожидается, что много вариантов будут отсеиваться как не удовлетворяющие контрактам, и большая асимптотика переборного алгоритма не будет сказываться на практике.
    \item В основной реализации требуется предоставить универсальный доступ к именованным атрибутам записей для кода пользовательских операций, аннотированных контрактами. Но ключевые операции Apache Beam нередко принимают и возвращают довольно комплексные структуры данных, для которых сложно обеспечить требуемое API \cite{beam-transforms}.
\end{itemize}

До применения ц-графов на практике требуется ещё решить следующие задачи:
\begin{itemize}
    \setlength\itemsep{-0.2em}
    \item Рассмотреть большое количество примеров из практики, поправить контракты и способ их записи, чтобы удобно было решать практические задачи.
    \item Перенести реализацию с прототипа на Python.
    \item Разработать функцию оценки, чтобы в соответствии с ней выбирать самый оптимальный конкретный граф.
\end{itemize}
