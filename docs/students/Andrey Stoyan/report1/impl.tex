\section{Вид реализации}

Реализация состоит из двух частей: основная, которую предполагается использовать на практике, и прототип, представляющий собой высокоуровневое описание модели контрактов и работы с ней.

\subsection{Основная реализация}

Основная реализация представлена в виде библиотеки на языке Python, позволяющей использовать ц-графы для задания вычислительных графов в Apache Beam Python API \cite{beam-py}.

Проект Apache Beam предоставляет универсальную модель задания вычислительных графов и позволяет запускать графы, написанные в этой модели, в различных рантаймах \cite{beam}. Например, на Apache Flink, который мы в качестве примера рассматривали выше.

Python был выбран как язык с динамической типизацией, потому что статическая мешала бы произвольной перестановке пользовательских операций.

Реализацию библиотеки можно найти в \href{https://github.com/flame-stream/calco/tree/master/calco}{GitHub репозитории}.

\subsection{Прототип}

Для реализации прототипа был выбран Haskell~--- чистый функциональный язык программирования с сильной статической системой типов \cite{haskell}.

В терминах типов удобно вводить нужные нам понятия, что и было сделано выше.
Статическая типизация предоставляет довольно высокие гарантии корректности.
А выразительность функционального языка позволяет естественно и довольно формально описывать работу с контрактами.

В прототипе не предполагалось реализовывать тела операций, аннотированных контрактами, поэтому статическая типизация не является проблемой в данном случае.

Реализацию прототипа можно найти в \href{https://github.com/flame-stream/calco/tree/master/calcohs}{GitHub репозитории проекта}.
