\section{Приложение А: развитие идеи} % TODO удалить нумерацию

Первоначальной идеей была следующая: задавать множество графов, решающих данную задачу как пару из типа требуемого результата вычислений и набора типизированных функций (операций) и констант (потоков)~--- контекста.
Тогда каждый конкретный граф задавался бы термом, имеющим заданный тип в данном контексте.
Имея алгоритм, перебирающий такие термы, мы бы получали множество конкретных графов, решающих нашу задачу.
При наличии функции оценки эффективности конкретного графа, можно было бы выбрать из множества самый эффективный.

Этот подход позволяет в задании пайплайна абстрагироваться от конкретных топологий графов, не указывая конкретные потоки данных.

Существует попытка реализации этого подхода: \url{https://github.com/solariq/yasm4u}.


